\documentclass[a4paper]{scrartcl}

\usepackage[ngerman]{babel}
\usepackage[utf8]{inputenc}
\usepackage[T1]{fontenc}
\usepackage{enumerate} % Für weitere Aufzählungstypen

\title{Aufzählungen}
\author{}
\date{}

\begin{document}
\maketitle

\section*{Aufzählung}
\begin{itemize}
    \item Foo
    \item Bar
\end{itemize}

\section*{Aufzählung mit Zahlen}
\begin{enumerate}
    \item Der erste
    \item Der zweite
\end{enumerate}

\section*{Selbstdefinierte Aufzählung mit enumerate Paket}
\begin{enumerate}[(a)]
    \item Lösung zu Teilaufgabe (a)
    \item Lösung zu Teilaufgabe (b)
\end{enumerate}

\section*{Beschreibung mit description}
\begin{description}
    \item[Gandalf] Ein ganz toller Zauberer
    \item[Lord Voldemort] Ein sehr, sehr böser Zauberer.
    Übrigens sehen descriptions auch immer noch gut aus, wenn der Text über mehrere Zeilen geht, deshalb ein wenig bla bla hier.
\end{description}
\end{document}
