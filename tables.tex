\documentclass[a4paper]{scrartcl}

\usepackage[ngerman]{babel}
\usepackage[utf8]{inputenc}
\usepackage[T1]{fontenc}
\usepackage{booktabs}

\title{Tabellen}
\author{}
\date{}

\begin{document}
\maketitle

\section*{tabular}
\begin{center}
% {l|ll} gibt an: Es gibt drei Spalten, zwischen der ersten und zweiten gibt es eine vertikale Trennlinie.
% Alle drei Spalten sind linksbündig. Weitere Werte sind c (zentriert) und r (rechtsbündig)
\begin{tabular}{l|ll}
Studiengang & WS 16/17 & WS 17/18  \\ \hline % hline erzeugt horizontale Trennlinie
Informatik & 904 & 941\\
MST & 916 & 879\\
SSE & 42 & 78
\end{tabular}
% Einzelne Spalten werden durch & getrennt
\end{center}

\section*{booktabs}
\begin{center}
\begin{tabular}{@{}lrr@{}}
% Tabellen: Übersichtlich halten, Linien vermeiden (max. 3 Stück) ==> booktabs Paket
\toprule
Studiengang & WS 16/17 & WS 17/18  \\
\midrule
Informatik & 904 & 941\\
MST & 916 & 879\\
SSE & 42 & 78\\
\bottomrule
\end{tabular}
\end{center}
\end{document}
