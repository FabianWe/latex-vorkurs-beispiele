\documentclass[a4paper]{scrartcl}

\usepackage[ngerman]{babel}
\usepackage[utf8]{inputenc}
\usepackage[T1]{fontenc}
\usepackage{mathtools} % oft wird stattdessen amsmath verwendet, mathtools kann bisschen mehr
\usepackage{amssymb} % Lädt mathematische Symbole

\title{Mathematische Formeln}
\author{}
\date{}

\begin{document}
\maketitle

\section*{Inline}
In einem Text kann inline eine Formel eingefügt werden, das geht ganz einfach \(a^2 + b^2 = c^2\).
% ^ sorgt dafür, dass das folgende Zeichen hochgestellt wird.

\section*{Symbole}
Im sogenannten \textit{math mode} können auch viele mathematische Symbole verwendet werden: \(\mathbb{N}, \varphi, \alpha\)

\section*{Mehrzeilige Formeln}
Für mehrzeilige Formeln verwendet man meist die align Umgebung.
% Der Stern entfernt die Nummerierung
\begin{align*}
(a + b)^2 &= (a + b) \cdot (a + b)\\
          &= a^2 + 2ab + b^2
\end{align*}
% Das & Zeichen sorgt dafür, dass die nachfolgenden Zeichen untereinander stehen
% Sollen mehrere Zeichen hochgestellt werden (nicht nur das nächste), so wird der entsprechende Block in { } eingefasst

Mit Nummerierung erhalten wir:
\begin{align}
(a + b)^2 &= (a + b) \cdot (a + b)\\
          &= a^2 + 2ab + b^2
\end{align}

Man kann auch Integrale und Summen einfach schreiben, ebenso sind Brüche möglich:
\begin{align*}
    \sum\limits_{i=0}^n i = \frac{n (n + 1)}{2}
\end{align*}
Ein Integral geht so:
\begin{align*}
    \int\limits_0^3 4x\ dx = 18
\end{align*}
% \ gefolgt von einem Leerzeichen erzwingt einen Abstand, ansonsten hängt es bisschen nah aneinander
Im Fließtext verwendet man hingegen eher kompakte Darstellungen, z.B. \(\int_0^3 4x\ dx = 18\) oder \(\sum_{i=0}^n a_i \coloneqq a_0 + a_1 + \dots + a_n \).
% \limits sorgt dafür, dass "i = 0" und "n" unter bzw. über dem Symbol erscheinen.
% _ sorgt dafür, dass das folgende Zeichen tiefergestellt wird

Verwendet man Klammern, kann das ganze auch schonmal etwas komisch aussehen, wie hier:
\begin{equation*}
    (\frac{a}{2} + \frac{b}{4})8
\end{equation*}
Aber auch dafür hat \LaTeX{} eine Lösung:
\begin{equation*}
    \left(\frac{a}{2} + \frac{b}{4}\right)8
\end{equation*}
Will man größere Brüche darstellen, kann man die dfrac Umgebung verwenden:
\begin{equation*}
    \dfrac{ \frac{a}{2} }{ b } \cdot 8
\end{equation*}
Matrizen usw. sind natürlich auch möglich.
\end{document}
