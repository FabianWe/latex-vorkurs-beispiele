\documentclass[a4paper]{scrartcl}

\usepackage[ngerman]{babel}
\usepackage[utf8]{inputenc}
\usepackage[T1]{fontenc}
\usepackage{csquotes} % kann das wörtliche zitieren erleichtern

\title{Referenzen und Zitieren mit bibtex}
\author{}
\date{}

\begin{document}
\maketitle

% Hier zitieren wir mit dem \cite Befehl
% in einem .bib file stehen mehrere Einträge, diese muss man nicht von Hand schreiben sondern kann sie z.B. aus google scholar kopieren: scholar.google.com
Alan Turing hat in seinem wegweisenden Artikel „On computable numbers, with an application to the Entscheidungsproblem“ \cite{turing1937computable} bewiesen, dass es Probleme gibt, für die kein Algorithmus existieren kann der diese löst.
Hierzu zählt zum Beispiel das bekannte Halteproblem.
Oder wie Turing selbst gesagt hat:
% Der folgende Befehl ist aus dem csquotes Paket, kann man nehmen, muss man aber nicht
\enquote{[\dots] what I shall prove is quite different from the well-known results of Gödel [\dots] I shall now show that there is no general method which tells whether a given formula \(U\) is provable in \(K\)}.

% Hier wird der Style festgelegt, es gibt da verschiedene
\bibliographystyle{apalike}
% Hier wird unsere Datei refs.bib geladen, die Endung .bib muss weggelassen werden
\bibliography{refs}
% Hinweis: Damit bibtex richtig funktioniert, muss pdflatex mehrfach gestartet werden.
% Die Reihenfolge ist wie folgt (jeweils auf die Datei die man kompilieren will losgelassen)
% pdflatex
% bibtex
% pdflatex
% pdflatex
\end{document}
